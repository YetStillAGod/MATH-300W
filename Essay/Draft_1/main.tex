\documentclass{article}
\usepackage[style=mla]{biblatex}
\usepackage{hyperref}
\addbibresource{bibliography.bib}

\begin{document}

\begin{abstract}
    This study investigates the independent emergence of the arithmetic triangle in Song Dynasty China (960--1279) and Renaissance Europe (14th--17th centuries). By analyzing primary texts and contextualizing discoveries within distinct sociopolitical frameworks, the essay challenges Eurocentric narratives of mathematical progress. Focus is placed on the bureaucratic algorithms of Jia Xian and Yang Hui versus the combinatorial empiricism of Petrus Apianus and Niccolò Tartaglia.
\end{abstract}

%------------------
% China’s Discovery
%------------------
\section{China’s Bureaucratic Mathematics: The Song Dynasty Triangle}

\subsection{Jia Xian and the Art of Root Extraction}
During the Song Dynasty’s bureaucratic reforms, mathematician Jia Xian (c. 1010--1070) developed the arithmetic triangle to solve cubic equations for land surveys. His lost treatise, \textit{Huangdi Jiuzhang Suanjie Xicao} (\textit{Detailed Solutions to the Nine Chapters}), emphasized iterative multiplication techniques. As reconstructed by Yang Hui, Jia Xian’s method required “coefficients arranged in ascending rows, each derived through additive multiplication” \cite{lam1994}. This process enabled precise calculation of land boundaries and tax quotas, reflecting the Song state’s reliance on mathematical standardization.

\subsection{Yang Hui’s Decimal Innovations}
In 1261, Yang Hui’s \textit{Xiangjie Jiuzhang Suanfa} (\textit{Detailed Analysis of the Nine Chapters}) formalized Jia Xian’s triangle with decimal fractions. The vertical diagram (Fig. \ref{fig:yanghui}) displayed coefficients for binomial expansions up to $(a + b)^8$, explicitly noting: 
\begin{quote}
    “Row nine yields 1, 8, 28, 56, 70, 56, 28, 8, 1—each number the sum of its two predecessors.”  
\end{quote}
Yang Hui applied these coefficients to granary volume calculations, integrating decimal alignment with the \textit{Nine Chapters}’ measurement systems. Unlike European counterparts, his triangle served administrative precision rather than theoretical exploration.

\subsection{Statecraft and the Civil Service}
The Song civil service examinations prioritized mathematical proficiency, particularly in the revised \textit{Nine Chapters}. Officials like Yang Hui systematized algorithms for infrastructure projects—canal dredging, flood control, and granary construction—where cubic root calculations proved essential. As \cite{martzloff1997} observes, this bureaucratic pragmatism framed mathematics as a governance tool rather than abstract pursuit.

%---------------------
% Europe’s Development
%---------------------
\section{Europe’s Mercantile Mathematics: From Astronomy to Dice Games}

\subsection{Apianus and the Printed Triangle}
The 1527 \textit{Practica auf Deutsch} by German astronomer Petrus Apianus featured Europe’s first printed arithmetic triangle (Fig. \ref{fig:apianus}). Designed to calculate Easter dates, Apianus’s horizontal “Rechenspiegel” (calculation mirror) simplified lunar cycle computations. His instructions advised: 
\begin{quote}
    “Sum adjacent numbers row by row to determine celestial intervals.”  
\end{quote}
This utilitarian design, devoid of algebraic context, catered to Renaissance Europe’s demand for accessible astronomical tables.

\subsection{Tartaglia’s Combinatorial Turn}
Niccolò Tartaglia’s 1556 \textit{General Trattato} repurposed the triangle for probability. Analyzing dice games, he noted: 
\begin{quote}
    “Row six’s coefficients (1, 5, 10, 10, 5, 1) enumerate the 32 possible outcomes of five dice throws.”  
\end{quote}
This application mirrored Europe’s burgeoning insurance and banking sectors, where risk assessment demanded combinatorial logic. Tartaglia’s work, as \cite{stedall2012} argues, epitomized the Renaissance shift toward empirical problem-solving.

\subsection{Print Culture and Mathematical Dissemination}
The printing press enabled mass production of mathematical texts like Apianus’s almanacs. Unlike China’s state-sponsored treatises, European mathematicians often published commercially, appealing to merchants and navigators. Navigational charts, insurance tables, and gambling manuals popularized the triangle’s combinatorial applications, reflecting what \cite{hacking1975} terms Europe’s “probabilistic turn.”

%------------------
% Comparative Analysis
%------------------
\section{Divergent Frameworks: Why Context Mattered}

\subsection{Algorithmic Precision vs. Empirical Utility}
China’s decimal-aligned triangle optimized bureaucratic tasks. Yang Hui’s inclusion of fractions like 3.75 in his granary calculations demonstrated seamless integration with state measurement systems. Europe’s integer-based triangle, by contrast, served probabilistic reasoning—Tartaglia’s dice tables prioritized whole-number outcomes for gambling odds. These differences, as \cite{restivo1992} notes, stemmed from China’s centralized governance versus Europe’s competitive markets.

\subsection{Epistemic Erasure and Rediscovery}
Despite Pascal’s 1654 systematization, Chinese contributions remained obscured until 20th-century scholarship. Jesuit missionaries like Matteo Ricci encountered Yang Hui’s work but dismissed it as “practical arithmetic” \cite{chen2005}. This dismissal, argues \cite{joseph2011}, exemplifies how colonial-era Eurocentrism marginalized non-Western innovations. Only Needham’s \textit{Science and Civilisation in China} (1959) reestablished Yang Hui’s legacy.

%------------------
% Conclusion
%------------------
\section{Conclusion}
The arithmetic triangle’s dual emergence underscores mathematics as a cultural artifact. Song China’s bureaucratic needs birthed decimal-aligned algorithms, while Renaissance Europe’s mercantilism fostered combinatorial empiricism. Recognizing these parallel trajectories dismantles myths of European exceptionalism, restoring agency to overlooked innovators like Jia Xian and Yang Hui.

\printbibliography

\newpage

\begin{enumerate}
    \item I think it is somewhat interesting with some insights.
    \item I will get more detailed and expand every section within this paper, and perhaps some changes in the bibliography.
    \item When I tried to find the primary sources, I found all in other languages and few of them have English translations, so I can only find secondary sources. Perhaps there is another way to get through this problem.
\end{enumerate}

\end{document}
