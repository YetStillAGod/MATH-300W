\documentclass{article}
\usepackage{amsmath, amssymb , amsthm, graphicx,mathtools}

\begin{document}

\section*{Question 1}

~
\subsection*{a}

~

\subsubsection*{i}

~

\begin{proof}
~
    \begin{enumerate}
        \item $\exists x,y\in \mathbb{Z}:ax+by=c$
        \item $a=\gcd(a,b)\cdot \dfrac{a}{\gcd(a,b)}$ and $b=\gcd(a,b)\cdot \dfrac{b}{\gcd(a,b)}$
        \item Substituting $a$ and $b$ into the equation:
        \begin{enumerate}
            \item $\gcd(a,b)\cdot \dfrac{a}{\gcd(a,b)}\cdot x+\gcd(a,b)\cdot \dfrac{b}{\gcd(a,b)}\cdot y=c$
            \item $c=\gcd(a,b)\left(\dfrac{a}{\gcd(a,b)}\cdot x+\dfrac{b}{\gcd(a,b)}\cdot y\right)$
        \end{enumerate}
        \item Since $x,y\in\mathbb{Z}$, $\dfrac{a}{\gcd(a,b)}\cdot x+\dfrac{b}{\gcd(a,b)}\cdot y\in\mathbb{Z}$
        \item So $\gcd(a,b)|c$
    \end{enumerate}
\end{proof}

~

\subsubsection*{ii}

~

\begin{proof}
    ~

    \begin{enumerate}
        \item $ax_1+by_1=c$
        \item Substituting $x_1$ to $x_1+mb$ and $y_1$ to $y_1-ma$:
        \begin{enumerate}
            \item $LHS=a(x_1+mb)+b(y_1-ma)=ax_1+mab+by_1-mab=ax_1+by_1$
            \item $ax_1+by_1=c$
        \end{enumerate}
        \item So $a(x_1+mb)+b(y_1-ma)=c$, and $(x_1+mb,y_1-ma)$ is a solution to the equation.
        \item if $(x_1,y_1)$ is a solution, $(x_1+mb,y_1-ma)$ is also a solution to the equation.
    \end{enumerate}
\end{proof}

~

\subsection*{b}


~

\begin{enumerate}
    \item Change the words into mathematical language: find $x$: $63x+7\equiv 0\pmod{23}$
    \item $63=23\times 2 +17$
    \item So the equation can be reduced to $17x+7\equiv 0\pmod{23}$
    \item $17x\equiv -7\pmod{23}$
    \item $-7\mod{23}=16\pmod{23}$
    \item So $17x\equiv 16\pmod{23}$
    \item $17^{-1}=\dfrac{1}{17}\equiv\dfrac{1}{-6}\equiv\dfrac{24}{-6}=-4\equiv 19\pmod{23}$
    \item So $17^{-1}\equiv 19\pmod{23}$
    \item Multiplying $19$ on both sides: $x\equiv 16\times 19=304\equiv5\pmod{23}$
    \item So the smallest solution is 5
\end{enumerate}

\newpage

\section*{Question 2}

~

\subsection*{a}

~

\begin{enumerate}
    \item \begin{enumerate}
        \item $N=24a+6$
        \item $N=98b+18$
    \end{enumerate}
    \item So $24a-98b=12$, $12a-49b=6$
    \item $49-12\times 4 =1$
    \item So $12\times -24 - 49\times -6=6$
    \item $(-24,-6)$ is a solution
    \item by 1 a ii,$(-24+49,-6+12)=(25,6)$ is a solution.
    \item $N=24\times 25+6=606=98\times 6+18$
    \item $N=606$ is the smallest positive solution.
\end{enumerate}

~

\subsection*{b}

~

\begin{enumerate}
    \item \begin{enumerate}
        \item $N=11a+2$
        \item $N= 13b+3$
        \item $N=18c+5$
    \end{enumerate}
    \item Use the first two equations: 
    \begin{enumerate}
        \item $11a-13b=1$
        \item $11\times 6-13\times 5=1$
        \item $(6,5)$ is a solution
        \item $(6+13d,5+11d)$ is also a solution
        \item $N=11(6+13d)+2=143d+68$
    \end{enumerate}
    \item Use the third equation: 
    \begin{enumerate}
        \item $ 143d-18c=-63$
        \item $ 143d\equiv -63\pmod{18}$
        \item $ 17n\equiv 9\pmod{18}$
        \item $17^{-1}\equiv 17\pmod{18}$
        \item $d\equiv 9\times 17=153\equiv 9\pmod{18}$
        \item $d=9\implies c=75$
        \item $(9,75)$ is a solution, which is also the smallest positive solution for $N$ since all the other smallers go to negative.
    \end{enumerate}
    \item $N=143\times 9+68=1355=18\times 75+5$
    \item $N=1355$ is the smallest positive solution.
\end{enumerate}

\newpage

\section*{Question 3}

~

\subsection*{a}

~

\begin{proof}
~
    \begin{enumerate}
        \item case 1 ($v$ is even):
        \begin{enumerate}
            \item $u_1=\dfrac{1}{2}uv(v^2+1)(v^2+3)$:
            \begin{enumerate}
                \item $u\in\mathbb{Z}$;
                \item $\frac{1}{2}v\in\mathbb{Z}$ since $v$ is even;
                \item $v^2+1\in\mathbb{Z}$;
                \item $v^2+3\in\mathbb{Z}$;
                \item So $u_1\in\mathbb{Z}$.
            \end{enumerate}
            \item $v_1=(v^2+2)\left[\dfrac{1}{2}(v^2+1)(v^2+3)-1\right]$:
            \begin{enumerate}
                \item $v_1=\dfrac{1}{2}(v^2+1)(v^2+3)(v^2+2)-(v^2+2)$
                \item $v^2+2$ is even, so $\dfrac{1}{2}(v^2+1)(v^2+3)(v^2+2)\in\mathbb{Z}$;
                \item $v^2+2\in\mathbb{Z}$;
                \item So $v_1\in\mathbb{Z}$.
            \end{enumerate}
            \item So $u_1,v_1\in\mathbb{Z}$
        \end{enumerate}
        \item case 2 ($v$ is odd):
        \begin{enumerate}
            \item $u_1=\dfrac{1}{2}uv(v^2+1)(v^2+3)$:
            \begin{enumerate}
                \item $u\in\mathbb{Z}$;
                \item $v\in\mathbb{Z}$;
                \item $v$ is odd, so $v^2+1$ is even, $\dfrac{1}{2}(v^2+1)\in\mathbb{Z}$;
                \item $v^2+3\in\mathbb{Z}$;
                \item So $u_1\in\mathbb{Z}$.
            \end{enumerate}
            \item $v_1=(v^2+2)\left[\dfrac{1}{2}(v^2+1)(v^2+3)-1\right]$:
            \begin{enumerate}
                \item $v$ is odd, so $v^2+1$ is even, $\dfrac{1}{2}(v^2+1)\in\mathbb{Z}$;
                \item $v^2+3\in\mathbb{Z}$;
                \item So $\dfrac{1}{2}(v^2+1)(v^2+3)-1\in\mathbb{Z}$;
                \item $v^2+2\in\mathbb{Z}$;
                \item So $v_1\in\mathbb{Z}$.
            \end{enumerate}
            \item So $u_1,v_1\in\mathbb{Z}$
        \end{enumerate}
        \item So $(u_1,v_1)$ is a pair of integers.
    \end{enumerate}
\end{proof}

~

\subsection*{b}

~

\begin{enumerate}
    \item test for small $x$:
    \item $x=1$:
    \begin{enumerate}
        \item $13-4=y^2$\\
        \item $y=3$\\
        \item $x=1$ stands.
    \end{enumerate}
    \item $(u,v)=(1,3)$
\end{enumerate}

~

\subsection*{c}

~

\begin{enumerate}
    \item $(u,v)=(1,3)$
    \item $u_1=\dfrac{1}{2}uv(v^2+1)(v^2+3)=\dfrac{1}{2}\cdot 1\cdot 3(3^2+1)(3^2+3)=180$
    \item $v_1=(v^2+2)\left[\dfrac{1}{2}(v^2+1)(v^2+3)-1\right]=(3^2+2)\left[\dfrac{1}{2}(3^2+1)(3^2+3)-1\right]=649$
    \item So $(u_1,v_1)=(180,649)$ is a solution to the equation.
\end{enumerate}

\newpage

\section*{Question 4}

~

\begin{proof}

~
    \begin{enumerate}
    \item Suppose $\square ABCD$ which diagonals intersect at $O$
    \item $AC\perp BD\implies AB^2+CD^2=AD^2+BC^2$:
    \begin{enumerate}
        \item $AC\perp BD$;
        \item So $\angle AOB=\angle BOC=\angle COD=\angle DOA=90^\circ$;
        \item By Pythagorean Theorem: 
        \begin{enumerate}
            \item $AB^2=AO^2+BO^2$
            \item $BC^2=BO^2+CO^2$
            \item $CD^2=CO^2+DO^2$
            \item $DA^2=DO^2+AO^2$
        \end{enumerate}
        \item $AB^2+CD^2=AO^2+BO^2+CO^2+DO^2$
        \item $AD^2+BC^2=BO^2+CO^2+DO^2+AO^2$
        \item So $AB^2+CD^2=AD^2+BC^2$
    \end{enumerate}
    \item $AB^2+CD^2=AD^2+BC^2\implies AC\perp BD$:
    \begin{enumerate}
        \item $\alpha\coloneqq\angle AOB=\angle COD, \beta\coloneqq \angle BOC=\angle DOA$
        \item By Law of Cosines:
        \begin{enumerate}
            \item $AB^2= AO^2+BO^2-2AO\cdot BO\cos \alpha$
            \item $BC^2= BO^2+CO^2-2BO\cdot CO\cos \beta$
            \item $CD^2= CO^2+DO^2-2CO\cdot DO\cos \alpha$
            \item $DA^2= DO^2+AO^2-2DO\cdot AO\cos \beta$
        \end{enumerate}
        \item By $AB^2+CD^2=AD^2+BC^2$:
        
        $AO^2+BO^2-2AO\cdot BO\cos \alpha+CO^2+DO^2-2CO\cdot DO\cos \alpha= BO^2+CO^2-2BO\cdot CO\cos \beta+DO^2+AO^2-2DO\cdot AO\cos \beta$
        \item $(AO\cdot BO+CO\cdot DO)\cos \alpha=(BO\cdot CO+DO\cdot AO)\cos \beta$
        \item $\alpha+\beta=180^\circ\implies \cos\alpha=-\cos\beta$
        \item $(AO\cdot BO+CO\cdot DO+BO\cdot CO+DO\cdot AO)\cos\alpha=0$
        \item $AO,BO,CO,DO>0\implies AO\cdot BO+CO\cdot DO+BO\cdot CO+DO\cdot AO\ne0 $
        \item So $\cos\alpha=0\implies \alpha =90^\circ$
        \item $AC\perp BD$
    \end{enumerate}
    \item $AC\perp BD\iff AB^2+CD^2=AD^2+BC^2$
\end{enumerate}
\end{proof}

\newpage

\section*{Question 5}

~

\subsection*{a}

~

\begin{proof}
    ~
    \begin{enumerate}
        \item $LHS>0$
        \item $\sqrt{\dfrac{a+\sqrt{a^2-b}}{2}}>\sqrt{\dfrac{a-\sqrt{a^2-b}}{2}}\implies RHS>0$
        \item So squaring does not affect the positivity of both sides
        
        i.e. $LHS^2=RHS^2\implies LHS=RHS$
        \item \begin{align*}
            &RHS^2\\
            =&\left(\sqrt{\dfrac{a+\sqrt{a^2-b}}{2}}\pm\sqrt{\dfrac{a-\sqrt{a^2-b}}{2}}\right)^2\\
            =&\dfrac{a+\sqrt{a^2-b}}{2}+\dfrac{a-\sqrt{a^2-b}}{2}\pm 2\sqrt{\dfrac{a+\sqrt{a^2-b}}{2}}\sqrt{\dfrac{a-\sqrt{a^2-b}}{2}}\\
            =&a\pm\sqrt{a^2-(a^2-b)}\\
            =&a\pm \sqrt{b}\\
        \end{align*}
        \item $LHS^2=a\pm\sqrt{b}$
        \item So $RHS^2=LHS^2$
        \item $RHS=LHS$
        \item $\sqrt{a\pm \sqrt{b}}=\sqrt{\dfrac{a+\sqrt{a^2-b}}{2}}\pm\sqrt{\dfrac{a-\sqrt{a^2-b}}{2}}$
    \end{enumerate}
\end{proof}

~

\subsection*{b}

~

\begin{enumerate}
    \item $a=17,b=240$
    \item $\sqrt{17+ \sqrt{240}}=\sqrt{\dfrac{17+\sqrt{17^2-240}}{2}}+\sqrt{\dfrac{17-\sqrt{17^2-240}}{2}}$
    \item $\sqrt{\dfrac{17+\sqrt{17^2-240}}{2}}+\sqrt{\dfrac{17-\sqrt{17^2-240}}{2}}=\sqrt{\dfrac{17+7}{2}}+\sqrt{\dfrac{17-7}{2}}$
    \item $\sqrt{\dfrac{17+7}{2}}+\sqrt{\dfrac{17-7}{2}}=\sqrt{12}+\sqrt{5}$
    \item So $\sqrt{17+ \sqrt{240}}=\sqrt{12}+\sqrt{5}$ 
\end{enumerate}

\newpage

\section*{Question 6}

~

\subsection*{a}

~

\begin{enumerate}
    \item $L(x)=kx+b$
    \item $k=\dfrac{f(x_2)-f(x_1)}{x_2-x_1}$
    \item $\dfrac{f(x_2)-f(x_1)}{x_2-x_1}\cdot x_1+b=f(x_1)$
    \item $b=f(x_1)-\dfrac{f(x_2)-f(x_1)}{x_2-x_1}\cdot x_1$
    \item $L(x)=\dfrac{f(x_2)-f(x_1)}{x_2-x_1}\cdot x+f(x_1)-\dfrac{f(x_2)-f(x_1)}{x_2-x_1}\cdot x_1$
    \item $L(x_3)=0$
    \item $x_3=\dfrac{\dfrac{f(x_2)-f(x_1)}{x_2-x_1}\cdot x_1-f(x_1)}{\dfrac{f(x_2)-f(x_1)}{x_2-x_1}}=\dfrac{f(x_2)x_1-f(x_1)x_2}{f(x_2)-f(x_1)}$
\end{enumerate}

~

\subsection*{b}

~

\begin{enumerate}
    \item $x_1=2,f(x_1)=8$
    \item $x_2=3,f(x_2)=-9$
    \item $x_3=\dfrac{-9\cdot 2-8\cdot 3}{-9-8}=\dfrac{42}{17}$
    \item $f(x_3)=-\dfrac{9144}{4913}$
    \item Use $x_1$ and $x_3$ to get $x_4$:
    \item $x_4=\dfrac{-\dfrac{9144}{4913}\cdot 2-8\cdot \dfrac{42}{17}}{-\dfrac{9144}{4913}-8}=\dfrac{1803}{757}$
    \item $f(x_4)=-\dfrac{100793169}{433798093}$
    \item Use $x_1$ and $x_4$ to get $x_5$:
    \item $x_5=\dfrac{-\dfrac{100793169}{433798093}\cdot 2-8\cdot \dfrac{1803}{757}}{-\dfrac{100793169}{433798093}-8}=\dfrac{29298426}{12357017}\approx2.370995\approx2.37$
    \item So the root is approximately 2.37.
\end{enumerate}
\end{document}