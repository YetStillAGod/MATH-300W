\documentclass{article}
\usepackage{amsmath, amsthm, amssymb}
\newtheorem{theorem}{Theorem}

\begin{document}
\section*{1}
~
The Similarities of the two is that they both have an accurate place-value system, that the position of one digit determines its values, like tens and hundreds. Counting boards also provides a visualized way of showing numbers and number calculations, like the educations now using visualized number blocks to teach arithmetic concepts. Differences are that counting boards require physical movement to do the calculation, while modern days are done on a paper and have the process mentally or using other calculation tools to assist. And counting boards does not have a symbol system like Arabic numbers do.

\newpage

\section*{2}

~

The method is to first have a initial guess to the square root, then use a series of steps to refine the initial guess in order to improve the accuracy of the guess.
\end{document}