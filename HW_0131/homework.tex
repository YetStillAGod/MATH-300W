\documentclass{article}
\usepackage[utf8]{inputenc}
\usepackage{setspace}
\usepackage{tikz}
\usetikzlibrary{positioning}
\usepackage{amsfonts}
\usepackage{amssymb}
\usepackage{amsmath}
\usepackage{amsthm}
\usepackage{systeme}
\usepackage{mathtools}
\usepackage{hyperref}
\usepackage{venndiagram}
\usepackage{pgfplots}
\usetikzlibrary{pgfplots.statistics}
\pgfplotsset{compat=newest}
\usepackage{logicproof}
\usepackage{mathrsfs}
\allowdisplaybreaks
\usepackage{mathpartir}
\usepackage{graphicx}

\begin{document}

\section*{Question 1}

~

\begin{proof}
    \begin{align*}
        &p(x)=x^n+a_{n-1}x^{n-1}+...+a_1x+a_0\\
        &\text{by substituting }x:\\
        &p(y)=(y-\frac{a_{n-1}}{n})^n+a_{n-1}(y-\frac{a_{n-1}}{n})^{n-1}+...+a_1(y-\frac{a_{n-1}}{n})+a_0\\
        &\text{To look at the }(n-1)\text{th term, it is sufficient to look at the }n\text{th term and }(n-1)\text{th term}\\
        &\text{Since all terms after }n-1\text{ cannot produce a term with degree }n-1\\
        \Rightarrow&p(y)=\sum_{i=0}^{n}\binom{n}{i}y^i(-\frac{a_{n-1}}{n})^{n-i}+a_{n-1}\sum_{i=0}^{n}\binom{n-1}{i}y^i(-\frac{a_{n-1}}{n})^{n-1-i}+...+a_1(y-\frac{a_{n-1}}{n})+a_0\\
        &\text{Looking at }(n-1)\text{th term}:\\
        &(y-\frac{a_{n-1}}{n})^n:\binom{n}{n-1}y^{n-1}(-\frac{a_{n-1}}{n})=-a_{n-1}y^{n-1}\\
        &a_{n-1}(y-\frac{a_{n-1}}{n})^{n-1}:a_{n-1}\binom{n-1}{n-1}y^{n-1}(-\frac{a_{n-1}}{n})^0=a_{n-1}y^{n-1}\\
        \Rightarrow&p(y)=b_ny^n+(-a_{n-1}+a_{n-1})y^{n-1}+b_{n-2}y^{n-2}+...+b_1y+b_0\\
        &p(y)=b_ny^n+b_{n-2}y^{n-2}+...+b_1y+b_0\\
        &\text{So the }(n-1)\text{th term is depressed}.
    \end{align*}
\end{proof}

\newpage

\section*{Question 2}

~

\subsection*{a}

~

\begin{align*}
    &x=-2\text{ is a root}\\
    \Rightarrow&(x^3+6x^2+x-14)\div(x+2)=x^2+4x-7\\
    &\text{The other two roots can be found by solving } x^2+4x-7=0\\
    &x=\frac{-4\pm\sqrt{16+28}}{2}=2\pm\sqrt{11}\\
    &x_1=-2,x_2=2+\sqrt{11},x_3=2-\sqrt{11}\\
\end{align*}

~

\subsection*{b}

~

\begin{align*}
    &x^3=3x^2+27x+41\\
    &x^3-3x^2=27x+41\\
    &x^3-3x^2+3x-1=30x+40\\
    &(x-1)^3=30x+40\\
    &y\coloneqq x-1\\
    &x=y+1\\
    &y^3=30y+70\\
    &y^3-30y-70=0\\
    &\text{Applying Cardano's formula}:\\
    &y=\sqrt[3]{-\frac{-70}{2}+\sqrt{\frac{(-30)^3}{27}+\frac{(-70)^2}{4}}}+\sqrt[3]{-\frac{-70}{2}-\sqrt{\frac{(-30)^3}{27}+\frac{(-70)^2}{4}}}\\
    &y=\sqrt[3]{35+\sqrt{-1000+1225}}+\sqrt[3]{35-\sqrt{-1000+1225}}=\sqrt[3]{50}+\sqrt[3]{20}\\
    \Rightarrow&x=\sqrt[3]{50}+\sqrt[3]{20}+1\\
\end{align*}

\newpage

\section*{Question 3}

~

\begin{proof}
    \begin{align*}
        &x^2+3x-36=0\\
        &p=3,q=-36\\
        &\text{Applying Cardano's formula}:\\
        &x=\sqrt[3]{-\frac{-36}{2}+\sqrt{\frac{3^3}{27}+\frac{36^2}{4}}}+\sqrt[3]{-\frac{-36}{2}-\sqrt{\frac{3^3}{27}+\frac{36^2}{4}}}\\
        &x=\sqrt[3]{18+\sqrt{325}}-\sqrt[3]{-18+\sqrt{325}}\\
        &x=\sqrt[3]{18+5\sqrt{13}}-\sqrt[3]{-18+5\sqrt{13}}\\
        &a+b\sqrt{13}=\sqrt[3]{18+5\sqrt{13}}\\
        &(a+b\sqrt{13})^3=18+5\sqrt{13}\\
        &a^3+3a^2b\sqrt{13}+39ab^2+13b^3\sqrt{13}=18+5\sqrt{13}\\
        &\begin{cases}
            a^3+39ab^2=18\\
            3a^2b+13b^3=5\\
        \end{cases}\\
        \Rightarrow&\begin{cases}
            a=\frac{3}{2}\\
            b=\frac{1}{2}\\
        \end{cases}\\
        \Rightarrow&\sqrt[3]{18+5\sqrt{13}}=\frac{3}{2}+\frac{1}{2}\sqrt{13}\\
        &\text{By the same method, }\sqrt[3]{-18+5\sqrt{13}}=-\frac{3}{2}+\frac{1}{2}\sqrt{13}\\
        \Rightarrow&x=\sqrt[3]{18+5\sqrt{13}}-\sqrt[3]{-18+5\sqrt{13}}=\frac{3}{2}+\frac{1}{2}\sqrt{13}-(-\frac{3}{2}+\frac{1}{2}\sqrt{13})=3\\
    \end{align*}
\end{proof}

\newpage

\section*{Question 4}

~

\subsection*{a}

~

\begin{align*}
    &(x^2+6)^2=x^4+12x^2+36\\
    &x^4+6x^2+8x+21=0\\
    &x^4+12x^2+36=6x^2-8x+15\\
    &(x^2+6)^2=6x^2-8x+15\\
\end{align*}

~

\subsection*{b}

~

\begin{align*}
    &(x^2+6+z)^2=x^4+12x^2+36+2x^2z+z^2+12z=(x^2+6)^2+2x^2z+z^2+12z\\
    &(x^2+6+z)^2=6x^2-8x+15+2x^2z+z^2+12z=(6+2z)x^2-8x+(z^2+12z+15)\\
\end{align*}

~

\subsection*{c}

~

\begin{align*}
    &(6+2z)x^2-8x+(z^2+12z+15)\\
    &\Delta=0\\
    \Rightarrow&(-8)^2-4(6+2z)(z^2+12z+15)=0\\
    &-8z^3-120z^2-408z-296=0\\
\end{align*}

~

\subsection*{d}

~

\begin{align*}
    &-8z^3-120z^2-408z-296=0\\
    &z^3+15z^2+51z+37=0\\
    &z\coloneqq y-\frac{15}{3}=y-5\\
    &(y-5)^3+15(y-5)^2+51(y-5)+37=0\\
    &y^3-24y+32=0\\
\end{align*}

~

\subsection*{e}

~

\begin{align*}
    &y=4\text{ is a root of }y^3-24y+32=0\\
    \Rightarrow&z=y-5=-1\\
    &(x^2+6+z)^2=(6+2z)x^2-8x+(z^2+12z+15)\\
    \Rightarrow&(x^2+5)^2=4x^2-8x+4\\
    &(x^2+5)^2=(2x-2)^2\\
    &(x^2+5)^2-(2x-2)^2=0\\
    &(x^2+2x+3)(x^2-2x+7)=0\\
    &(x+1+\sqrt{2}i)(x+1-\sqrt{2}i)(x-1+\sqrt{6}i)(x-1-\sqrt{6}i)=0\\
    \Rightarrow&\begin{cases}
        x_1=-1-\sqrt{2}i\\
        x_2=-1+\sqrt{2}i\\
        x_3=1-\sqrt{6}i\\
        x_4=1+\sqrt{6}i\\
    \end{cases}
\end{align*}
\end{document}
